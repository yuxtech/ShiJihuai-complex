\chapter{多复变数全纯函数与全纯映射\label{chap9}}
前面八章讨论的是一个复变数全纯函数的性质及其应用,从全纯函数的积分表示和级数表示到几何理论,它构成一个完美的体系.我们自然要问,如果复变数增加到两个或$n$($n>1$)个,这时候全纯函数是否还具有前面所说的那些性质呢? 1906年,Hartogs发现在$n$个复变数的空间中,存在这样一种域,这种域上的每一个全纯函数都可全纯开拓到比它更大的域上去,这种现象在复平面的域中是不会发生的.后人把这种现象称为Hartogs现象.从Hartogs现象马上可以推出,多个复变数的全纯函数的零点一定不是孤立的,这是又一个与单复变数全纯函数根本不同的性质!当然,只有在不发生Hartogs现象的域上研究函数论才是有意义的.我们把不发生Hartogs现象的域称为全纯域.因此,寻找刻画全纯域的特征就是一个十分重要的问题.围绕这个问题以及其他有关的问题,在多复变的研究中引进了一系列新概念和新方法,使多复变逐渐成为一门新的独立的学科.时至今日,多复变已经成为当代数学研究的主流方向之一.

本章将对多复变函数的若干性质作一个十分简单的介绍,目的是使读者看到多复变函数论和单复变函数论的一些本质区别,从而产生对多复变函数论的兴趣.

\section{多复变数全纯函数的定义\label{sec9.1}}
我们用$\MC^n$记$n$个坐标都是复数的$n$维向量的全体,即
\[\MC^n=\{z=(z_1,\cdots,z_n):z_j\in\MC,j=1,\cdots,n\}.\]
设$z=(z_1,\cdots,z_n),w=(w_1,\cdots,w_n)$是$\MC^n$中的两个点,$\lambda\in\MC$,定义它们的加法和数乘如下:
\begin{align*}
&z+w=(z_1+w_1,\cdots,z_n+w_n),\\
&\lambda z=(\lambda z_1,\cdots,\lambda z_n).
\end{align*}
在这样的定义下,$\MC^n$是复数域上的线性空间. $\MC^n$中向量$z$的长度定义为
\[|z|=\big(|z_1|^2+\cdots+|z_n|^2\big)^{\frac12}.\]

正像$\MC$可以看成$\MR^2$一样,$\MC^n$也可以看成$\MR^{2n}$.

$\MC^n$中的连通开集$\Omega$称为域. 当$\Omega$有界时,就称$\Omega$为有界域.
下面两类简单的有界域值得我们特别注意:

设$a=(a_1,\cdots,a_n)\in\MC^n,r=(r_1,\cdots,r_n),r_j>0,j=1,\cdots,n$. 称
\[P(a,r)=\{(z_1,\cdots,z_n):|z_j-a_j|<r_j,j=1,\cdots,n\}\]
为以$a$为中心、$r$为半径的\textbf{多圆柱}\index{D!多复变函数!多圆柱}. 特别地,当$a=0,r_j=1,j=1,\cdots,n$时,称之为\textbf{单位多圆柱}\index{D!多复变函数!单位多圆柱},记为$U^n$,即
\[U^n=\{(z_1,\cdots,z_n):|z_j|<1,j=1,\cdots,n\}.\]
显然,当$n=1$时,它就是单位圆盘.

以$a=(a_1,\cdots,a_n)$为中心、$\rho>0$为半径的\textbf{球}\index{D!多复变函数!球}是指
\[B(a,\rho)=\bigg\{(z_1,\cdots,z_n):\sum_{j=1}^n|z_j-a_j|^2<\rho^2\bigg\}.\]
特别地,当$a=0,\rho=1$时,称之为\textbf{单位球}\index{D!多复变函数!单位球}
,记为$B_n$,即
\[B_n=\bigg\{(z_1,\cdots,z_n):\sum_{j=1}^n|z_j|^2<1\bigg\}.\]
当$n=1$时,它也是单位圆盘.

$U^n$和$B_n$都是单位圆盘在$\MC^n$中的推广,但它们不是全纯等价的,即不存在双全纯映射把$B_n$映为$U^n$,这是著名的\textbf{Poincar\'e定理}\index{D!定理!Poincar\'e定理}.它说明单复变中的Riemann映射定理在多复变中是不成立的.

为了引进全纯函数的概念,我们要讨论多重幂级数的性质.先从多重级数讲起.

给定依赖两个指标的数列$\{a_{jk}\}$,称$\sum_{j,k=1}^\infty a_{jk}$为\textbf{二重级数}\index{E!二重级数},数$S_{m,n}=\sum_{j=1}^m\sum_{k=1}^n a_{jk}$称为它的部分和. 如果$\lim_{\substack{m\to\infty\\n\to\infty}}S_{m,n}=S$存在,就说上述二重级数是收敛的,$S$是它的和.

用同样的方法可以定义一般多重级数收敛的概念.

级数
\[\sum_{\alpha_1,\cdots,\alpha_n=0}^\infty c_{\alpha_1\cdots\alpha_n}(z_1-a_1)
^{\alpha_1}\cdots(z_n-a_n)^{\alpha_n}\]
称为\textbf{$n$重幂级数}\index{N!$n$重级数},它在点$b=(b_1,\cdots,b_n)$处收敛是指$n$重级数
\[\sum_{\alpha_1,\cdots,\alpha_n=0}^\infty c_{\alpha_1\cdots\alpha_n}(b_1-a_1)
^{\alpha_1}\cdots(b_n-a_n)^{\alpha_n}\]
收敛.

关于多重幂级数,也有类似与单变数中的Abel定理. 为简单起见,我们讨论$a=0$的情形.
\begin{prop}\label{prop9.1.1}
如果$n$重幂级数
\begin{equation}\label{eq9.1.1}
\sum_{\alpha_1,\cdots,\alpha_n=0}^\infty c_{\alpha_1\cdots\alpha_n}z_1^{\alpha_1}
\cdots z_n^{\alpha_n}
\end{equation}
在点$b=(b_1,\cdots,b_n)$处收敛,这里,$b_j\ne0,j=1,\cdots,n$. 那么它在闭多圆柱
\[\bar{P(0,r)}=\{(z_1,\cdots,z_n):|z_j|\le r_j,j=1,\cdots,n\}\]
中绝对且一致收敛,这里,$r_j<|b_j|,j=1,\cdots,n$.
\end{prop}
\begin{proof}
因为幂级数$\sum_{\alpha_1,\cdots,\alpha_n=0}^\infty c_{\alpha_1\cdots\alpha_n}b_1^{\alpha_1}
\cdots b_n^{\alpha_n}$收敛,所以存在常数$M$,使得对任意$\alpha_1,\cdots,\alpha_n\ge0$,有
\[|c_{\alpha_1\cdots\alpha_n}|\le\frac M{|b_1|^{\alpha_1}\cdots|b_n|^{\alpha_n}}.\]
故当$|z_j|\le r_j<|b_j|$($j=1,\cdots,n$)时,有
\[|c_{\alpha_1\cdots\alpha_n}z_1^{\alpha_1}
\cdots z_n^{\alpha_n}|\le M\bigg|\frac{z_1}{b_1}\bigg|^{\alpha_1}\cdots
\bigg|\frac{z_n}{b_n}\bigg|^{\alpha_n}
\le M\bigg(\frac{r_1}{|b_1|}\bigg)^{\alpha_1}\cdots
\bigg(\frac{r_n}{|b_n|}\bigg)^{\alpha_n},\]
于是
\begin{align*}
  \sum_{\alpha_1,\cdots,\alpha_n=0}^\infty |c_{\alpha_1\cdots\alpha_n}z_1^{\alpha_1}
  \cdots z_n^{\alpha_n}|&\le M\sum_{\alpha_1=0}^\infty \bigg(\frac{r_1}{|b_1|}\bigg)^{\alpha_1}\cdots\sum_{\alpha_n=0}^\infty
  \bigg(\frac{r_n}{|b_n|}\bigg)^{\alpha_n}\\
  & = M\bigg(1-\frac{r_1}{|b_1|}\bigg)^{-1} \cdots\bigg(1-\frac{r_n}{|b_n|}\bigg)^{-1}. \qedhere
\end{align*}
\end{proof}

为简化记号,我们采用下面的习惯记法:对于有序数组$\alpha=(\alpha_1,\cdots,\alpha_n)$,其中,每个$\alpha_j$都是非负整数,记
\begin{align*}
&|\alpha|=\alpha_1+\cdots+\alpha_n,\\
&\alpha!=\alpha_1!\cdots\alpha_n!,\\
&z^\alpha=z_1^{\alpha_1}\cdots z_n^{\alpha_n},
\end{align*}
其中,$z=(z_1,\cdots,z_n)$. 这样,幂级数 \eqref{eq9.1.1} 就可简记为$\sum_\alpha c_\alpha z^\alpha$或者$\sum_{\alpha\ge0}c_\alpha z^\alpha$.

现在可以给出多复变数全纯函数的概念了.
\begin{definition}\label{def9.1.2}
设$\Omega$是$\MC^n$中的域,$f:\Omega\to \MC$是定义在$\Omega$上的一个复值函数.如果对每一点$a\in\Omega$,存在多圆柱$P(a,\rho)\subset\Omega$和幂级数$\sum_\alpha c_\alpha(z-a)^\alpha$,使得
\begin{equation}\label{eq9.1.2}
f(z)=\sum_\alpha c_\alpha(z-a)^\alpha
\end{equation}
在$P(a,\rho)$中成立,则称$f$为$\Omega$中的\textbf{全纯函数}\index{Q!全纯函数}.
\end{definition}

与单复变中一样,我们用$H(\Omega)$记$\Omega$上全纯函数的全体.

设$f$在$a$点附近全纯,那么$f$在$a$点附近可以用幂级数 \eqref{eq9.1.2} 表示.如果把幂级数 \eqref{eq9.1.2} 写成
\[f(z)=\sum_{\alpha_1=0}^\infty\bigg\{\sum_{\alpha_2=0}^\infty\cdots
\sum_{\alpha_n=0}^\infty c_{\alpha_1\cdots\alpha_n}(z_2-a_2)^{\alpha_2}\cdots(z_n-a_n)^{\alpha_n}\bigg\}
(z_1-a_1)^{\alpha_1},\]
那么当$z_2,\cdots,z_n$固定时,上式是$f(z_1,z_2,\cdots,z_n)$关于$z_1$在$a_1$附近的幂级数展开式,因此它是$z_1$的全纯函数.一般来说,如果$f$在$a$点附近全纯,那么当$z_1,\cdots,z_{j-1},z_{j+1}$, $\cdots,z_n$固定时,$f$便是单变数$z_j$的全纯函数,因而有Cauchy--Riemann方程
\begin{equation}\label{eq9.1.3}
\pp f{\bar{z_j}}=0,\;j=1,\cdots,n.
\end{equation}
这$n$个方程称为$f$的\textbf{Cauchy--Riemann方程组}\index{C!Cauchy--Riemann方程组}.

一个自然的问题是,如果 \eqref{eq9.1.3} 式在$a$点附近成立,能否断言$f$在$a$点附近全纯呢?答案是肯定的,但证明起来很困难,这里略去它的证明.
\begin{theorem}[(\textbf{Hartogs})]\label{thm9.1.3}\index{D!定理!Hartogs定理}
设$\Omega$是$\MC^n$中的域,$f:\Omega\to\MC$是定义在$\Omega$上的函数. 对于$a\in\MC^n$,定义$\MC$中的域$\Omega_{j,a}$及$\Omega_{j,a}$上的函数$f_{j,a}$如下:
\begin{align*}
&\Omega_{j,a}=\{z\in\MC:(a_1,\cdots,a_{j-1},z,a_{j+1},\cdots,a_n)\in\Omega\},\\
&f_{j,a}(z)=f(a_1,\cdots,a_{j-1},z,a_{j+1},\cdots,a_n).
\end{align*}
如果对任意的$a\in\MC^n$及$j=1,\cdots,n,f_{j,a}\in H(\Omega_{j,a})$,那么$f\in H(\Omega)$.
\end{theorem}

设$f\in H(\Omega)$,根据上面的讨论,当$z_1,\cdots,z_{j-1},z_{j+1},\cdots,z_n$固定时,$f$是$z_j$的全纯函数,因而对它的幂级数表达式可以逐项求导数,从幂级数 \eqref{eq9.1.2} 可得
\[\frac{\partial^{\alpha_1+\cdots+\alpha_n}f(z)}{\partial z_1^{\alpha_1}
\cdots\partial z_n^{\alpha_n}}\bigg|_{z=a}=\alpha!c_\alpha.\]
如果记
\[(\DD^\alpha f)(a)=\frac{\partial^{\alpha_1+\cdots+\alpha_n}f(z)}{\partial z_1^{\alpha_1}
\cdots\partial z_n^{\alpha_n}}\bigg|_{z=a},\]
那么$f$在$a$点的幂级数展开式 \eqref{eq9.1.2} 可写为
\[f(z)=\sum_\alpha\frac{(\DD^\alpha f)(a)}{\alpha!}(z-a)^\alpha.\]

类似于第 \ref{chap4} 章定理 \ref{thm4.3.7} 的唯一性定理在多复变中是不成立的. 例如,函数$f(z_1,z_2)=z_1z_2$在双圆柱$U^2=\{z=(z_1,z_2):|z_1|<1,|z_2|<1\}$中全纯,点列$\bigg\{\bigg(0,\frac1k\bigg),k=2,3,\cdots\bigg\}$以$(0,0)$为极限点,且$f\bigg(0,\frac1k\bigg)=0$,但$f$在双圆柱中并不恒等于零. 在多复变中,有下列形式的唯一性定理:
\begin{theorem}\label{thm9.1.4}
设$\Omega$是$\MC^n$中的域,$f\in H(\Omega)$. 如果$f$在非空开集$E$上恒等于零,那么$f$在$\Omega$上恒等于零.
\end{theorem}
\begin{proof}
令
\begin{align*}
&K=\{z\in\Omega:(D^\alpha f)(z)=0\,\mbox{对所有$\alpha=(\alpha_1,\cdots,\alpha_n)$成立}\},\\
&K_\alpha=\{z\in\Omega:(D^\alpha f)(z)=0\,\mbox{对某个$\alpha=(\alpha_1,\cdots,\alpha_n)$成立}\}.
\end{align*}
由假定,$E\subset K$,所以$K$不是空集. 显然
\[K=\bigcap_\alpha K_\alpha.\]
因为$\DD^\alpha f$是连续函数,所以$K_\alpha$是闭集,因而$K$也是闭集.任取$a\in K$,因为$f$在$\Omega$中全纯,故存在多圆柱$P(a,r)\subset \Omega$,使得
\[f(z)=\sum_\alpha\frac{(D^\alpha f)(a)}{\alpha!}(z-a)^\alpha=0\]
在$P(a,r)$中成立,因而$P(a,r)\subset K$,这说明$K$是一个开集. 由于$\Omega\backslash K$也是开集,等式
\[\Omega=K\cup(\Omega\backslash K)\]
与$\Omega$的连通性矛盾.因为$K$不是空集,所以只能$\Omega\backslash K$是空集,即$K=\Omega$.由此即知$f$在$\Omega$上恒等于零.
\end{proof}

作为唯一性定理的应用,我们可以证明下面的\textbf{开映射定理}\index{D!定理!开映射定理}:
\begin{theorem}\label{thm9.1.5}
设$\Omega$是$\MC^n$中的域,$f$是$\Omega$上非常数的全纯函数,那么$f$把$\Omega$中的开集映成$\MC$中的开集.
\end{theorem}
\begin{proof}
$n=1$时定理是成立的(见定理 \ref{thm4.4.6}).现设$n>1$.任取开集$E\subset\Omega$,我们要证明$f(E)$是$\MC$中的开集.为此任取$w\in f(E)$,则必存在$a\in E$,使得$f(a)=w$.令$Q$是$a$的一个凸邻域(例如可以取$Q$为包含$a$的一个多圆柱),$Q\subset E$,由定理 \ref{thm9.1.4} 知道,$f$在$Q$上不能恒等于$f(a)$,因而能找到$b\in Q$,使得$f(a)\ne f(b)$.令
\[D=\{\lambda\in\MC:a+\lambda(b-a)\in Q\},\]
显然$\lambda=0,\lambda=1$都属于$D$,因而$D$不是空集. 由于$Q$是开集,所以$D$也是$\MC$中的开集.令
\[g(\lambda)=f\big(a+\lambda(b-a)\big),\lambda\in D.\]
容易知道它是$D$上的全纯函数,且
\[g(0)=f(a)\ne f(b)=g(1),\]
即$g$不是常函数. 于是,由定理 \ref{thm4.4.6} 即知$g(D)$是$\MC$中的开集,且$w=f(a)=g(0)\in g(D)$.另一方面,$g(D)\subset f(Q)\subset f(E)$,因而$f(E)$是$\MC$中的开集.
\end{proof}

利用开映射定理又可得到下面的\textbf{最大模原理}\index{D!定理!最大模原理}:
\begin{theorem}\label{thm9.1.6}
设$\Omega$是$\MC^n$中的域,$f$是$\Omega$上非常数的全纯函数,那么$f$的模不可能在$\Omega$的内点达到最大值.
\end{theorem}

这个定理的证明与定理 \ref{thm4.5.1} 的证明一样,留给读者作为练习.
\begin{xiti}
\item 设$\Omega$是$\MC^n$中的域,$f\in H(\Omega)$.如果存在$a\in\Omega$,使得$(\DD^\alpha f)(a)=0$对所有多重指标$\alpha=(\alpha_1,\cdots,\alpha_n)$成立,证明在$\Omega$上$f(z)\equiv0$.
\item 设$\Omega$是$\MC^n$中的域,$f_1,\cdots,f_m\in H(\Omega)$. 如果$\sum_{j=1}^m|f_j(z)|^2$在$\Omega$上为一常数,证明$f_1,\cdots,f_m$都是常数.
\item 设$\Omega$是$\MC^n$中的域,$f,g\in H(\Omega)$. 如果$f(z)g(z)\equiv0$在$\Omega$上成立,证明$f$或$g$必在$\Omega$上恒等于零.
\item 证明定理 \ref{thm9.1.6}.
\end{xiti}

\section{多圆柱的Cauchy积分公式\label{sec9.2}}
在单复变中,Cauchy积分公式起了十分重要的作用,对于不同的域,Cauchy积分公式具有相同的形式.但在多复变中,情况要复杂得多,对于不同的域,有不同的Cauchy积分公式.下面先给出\textbf{多圆柱上的Cauchy积分公式}\index{G!公式!多圆柱上的Cauchy积分公式}.
\begin{theorem}\label{thm9.2.1}
设$\Omega$是$\MC^n$中的域,$f\in H(\Omega)$. 如果$\bar{P(a,r)}\subset \Omega$,则对$z\in P(a,r)$,有
\begin{equation}\label{eq9.2.1}
f(z)=\frac1{(2\pi\ii)^n}\int\limits_{|\zeta_1-a_1|=r_1}\cdots\int\limits_{|\zeta_n-a_n|=r_n}
\frac{f(\zeta_1,\cdots,\zeta_n)\dif \zeta_1\cdots\dif \zeta_n}{(\zeta_1-z_1)\cdots(\zeta_n-z_n)}.
\end{equation}
\end{theorem}
\begin{proof}
当$n=1$时,这是熟知的圆盘上的Cauchy积分公式.今设定理对$n-1$个变数的全纯函数成立.分别在圆周$|\zeta_2-a_2|=r_2,\cdots,|\zeta_n-a_n|=r_n$上固定$\zeta_2,\cdots,\zeta_n$,则$f(z_1,\zeta_2,\cdots,\zeta_n)$是圆盘$|z_1-a_1|\le r_1$上的全纯函数,由单复变的Cauchy积分公式得
\begin{equation}\label{eq9.2.2}
f(z_1,\zeta_2,\cdots,\zeta_n)=\frac1{2\pi\ii}\int\limits_{|\zeta_1-a_1|=r_1}
\frac{f(\zeta_1,\zeta_2,\cdots,\zeta_n)}{\zeta_1-z_1}\dif \zeta_1.
\end{equation}
固定$z_1$,把$f(z_1,z_2,\cdots,z_n)$看成$z_2,\cdots,z_n$的函数,用归纳法的假定,再利用 \eqref{eq9.2.2} 式,即得
\begin{align*}
  f(z_1,z_2,\cdots,z_n)&=\frac1{(2\pi\ii)^{n-1}}\int\limits_{|\zeta_2-a_2|=r_2}\cdots\int\limits_{|\zeta_n-a_n|=r_n}
  \frac{f(z_1,\zeta_2,\cdots,\zeta_n)\dif \zeta_2\cdots
  \dif \zeta_n}{(\zeta_2-z_2)\cdots(\zeta_n-z_n)}\\
  & = \frac1{(2\pi\ii)^n}\int\limits_{|\zeta_1-a_1=r_1} \cdots\int\limits_{|\zeta_n-a_n|=r_n}
  \frac{f(\zeta_1,\cdots,\zeta_n)\dif \zeta_1\cdots
  \dif \zeta_n}{(\zeta_1-z_1)\cdots(\zeta_n-z_n)} \qedhere
\end{align*}
\end{proof}

如果记$D_j=\{z_j\in\MC:|z_j-a_j|<r_j\},j=1,2,\cdots,n$,那么多圆柱$P(a,r)$是这$n$个圆盘的拓扑积。它的边界$\partial P$由若干部分组成,例如,$\partial D_1\times D_2\times\cdots\times D_n,\partial D_1\times\partial D_2\times D_3\times\cdots\times D_n,\cdots,\partial D_1\times\partial D_2\times\cdots\times\partial D_n$都是它的边界的组成部分.其中,最低维的那一部分
\[\partial D_1\times\cdots\times\partial D_n=\{(z_1,\cdots,z_n):|z_j-a_j|=r_j.j=1,\cdots,n\}\]
称为$P(a,r)$的\textbf{特征边界}\index{T!特征边界},记为$\partial_0P$.

多圆柱的Cauchy积分公式 \eqref{eq9.2.1} 的积分区域不是$P(a,r)$的全部边界,而只是它的边界的一部分——特征边界.这是多复变与单复变的一个重要区别.在单复变中,Cauchy积分公式的积分是在全部边界上进行的.

如果在Cauchy积分公式 \eqref{eq9.2.1} 中对$z$求导数,可得
\begin{equation}\label{eq9.2.3}
(\DD^\alpha f)(a)=\frac{\alpha!}{(2\pi\ii)^n}
\int\limits_{\partial_0P}
\frac{f(\zeta)\dif \zeta_1\cdots\dif \zeta_n}{(\zeta_1-a_1)^{\alpha_1+1}\cdots(\zeta_n-a_n)^{\alpha_n+1}}.
\end{equation}
由此可以得到下面的\textbf{Cauchy不等式}\index{B!不等式!Cauchy不等式}.
\begin{theorem}\label{thm9.2.2}
设$\Omega$是$\MC^n$中的域,$\bar{P(a,r)}\subset\Omega$. 如果$f\in H(\Omega)$,记$M=\sup\{|f(\zeta)|:\zeta\in\partial_0P\}$,那么对任意多重指标$\alpha=(\alpha_1,\cdots,\alpha_n)$,有
\[|(\DD^\alpha f)(a)\le M\frac{\alpha!}{r^\alpha}.\]
\end{theorem}
\begin{proof}
从等式 \eqref{eq9.2.3} 即得
\begin{align*}
  |(\DD^\alpha f)(a)|&\le\frac{\alpha!}{(2\pi)^n}
  \int\limits_{\partial_0P}
  \frac{|f(\zeta)|\,|\dif \zeta_1|\cdots|\dif \zeta_n|}{|\zeta_1-a_1|^{\alpha_1+1}\cdots|\zeta_n-a_n|^{\alpha_n+1}}\\
  & \le \frac{\alpha!}{(2\pi)^n}\frac M{r_1^{\alpha_1+1}\cdots r_n^{\alpha_n+1}} (2\pi)^nr_1\cdots r_n\\
  & = M\frac{\alpha!}{r^\alpha}. \qedhere
\end{align*}
\end{proof}

当$n=1$时,这就是定理 \ref{thm3.5.1}.

我们也有类似于引理 \ref{lemma4.1.8} 的结果.
\begin{theorem}\label{thm9.2.3}
设$\Omega$是$\MC^n$中的域,$f\in H(\Omega)$. 如果紧集$K$及其邻域$G$满足条件$K\subset G\subset\subset \Omega$,那么有不等式
\[\sup\{|(\DD^\alpha f)(z)|:z\in K\}\le C\sup\{|f(z)|:z\in G\},\]
这里,$C$是与$K,G$及$\alpha$有关的常数.
\end{theorem}
\begin{proof}
因为$\rho=d(K,\partial G)>0$,所以以$K$中任何点$a$为中心、$\rho$为半径的多圆柱
\[P=\{(z_1,\cdots,z_n):|z_j-a_j|<\rho,j=1,\cdots,n\}\]
都含在$G$中.于是由Cauchy不等式,有
\[|(\DD^\alpha f)(a)|\le\sup_{\zeta\in\partial_0P}|f(\zeta)|\frac{\alpha!}{\rho^{|\alpha|}}
\le C\sup\{|f(z)|:z\in G\}.\]
由此即得所要证的不等式.
\end{proof}

在多复变中,也有类似于单复变中的\textbf{Weierstrass定理}\index{D!定理!Weierstrass定理}.
\begin{theorem}\label{thm9.2.4}
设$\Omega$是$\MC^n$中的域,$\{f_k\}$是$\Omega$上的一列全纯函数,如果它在$\Omega$上内闭一致收敛于$f$,那么$f\in H(\Omega)$,而且对任意多重指标$\alpha=(\alpha_1,\cdots,\alpha_n)$,$\DD^\alpha f_k$在$\Omega$上内闭一致收敛于$\DD^\alpha f$.
\end{theorem}
\begin{proof}
对任意$a\in\Omega$,我们证明$f$在$a$附近能展开成幂级数. 适当选取$\rho=(\rho_1,\cdots,\rho_n)$,使得$\bar{P(a,\rho)}\subset\Omega$. 对$f_k$用Cauchy积分公式,得
\[f_k(z)=\frac1{(2\pi\ii)^n}\int\limits_{\partial_0P}\frac{f_k(\zeta)\dif \zeta_1\cdots\dif \zeta_n}{(\zeta_1-z_1)\cdots(\zeta_n-z_n)},\;z\in P(a,\rho).\]
由于当$k\to\infty$,$f_k$在$\partial_0P$上一致收敛于$f$,故在上式中令$k\to\infty$,即得
\begin{equation}\label{eq9.2.4}
f(z)=\frac1{(2\pi\ii)^n}\int\limits_{\partial_0P}\frac{f(\zeta)\dif \zeta_1\cdots\dif \zeta_n}{(\zeta_1-z_1)\cdots(\zeta_n-z_n)},\;z\in P(a,\rho).
\end{equation}
利用证明定理 \ref{thm4.3.1} 时用过的方法,有
\begin{align*}
\frac1{\zeta_j-z_j}&=\frac1{(\zeta_j-a_j)-(z_j-a_j)}
=\frac1{\zeta_j-a_j}\bigg(1-\frac{z_j-a_j}{\zeta_j-a_j}\bigg)^{-1}\\
&=\frac1{\zeta_j-a_j}\sum_{\alpha_j=0}^\infty\bigg(\frac{z_j-a_j}{\zeta_j-a_j}\bigg)
^{\alpha_j},
\end{align*}
于是
\[\frac1{(\zeta_1-z_1)\cdots(\zeta_n-z_n)}=\sum_{\alpha_1=0}^\infty\cdots
\sum_{\alpha_n=0}^\infty\frac{(z_1-a_1)^{\alpha_1}\cdots(z_n-a_n)^{\alpha_n}}
{(\zeta_1-a_1)^{\alpha_1+1}\cdots(\zeta_n-a_n)^{\alpha_n+1}}.\]
上述级数对于$\zeta\in\partial_0P$是一致收敛的,代入 \eqref{eq9.2.4} 式即得
\[f(z)=\sum_{\alpha\ge0}\bigg\{\frac1{(2\pi\ii)^n}\int\limits_{\partial_0P}
\frac{f(\zeta)\dif \zeta_1\cdots\dif \zeta_n}
{(\zeta_1-a_1)^{\alpha_1+1}\cdots(\zeta_n-a_n)^{\alpha_n+1}}\bigg\}(z-a)^\alpha.\]
把上面花括弧中的数记为$c_\alpha$,即得
\[f(z)=\sum_{\alpha\ge0}c_\alpha(z-a)^\alpha.\]
这就证明了$f\in H(\Omega)$.

对于任意紧集$K\subset D$,取其邻域$G$,使得$K\subset G\subset\subset \Omega$.因为$\bar G$是紧的,故对任意$\varepsilon>0$,存在$k_0$,当$k>k_0$时,有
\[\sup\{|f_k(z)-f(z)|:z\in\bar G\}<\varepsilon.\]
于是,由定理 \ref{thm9.2.3} 得
\[\sup\{|\DD^\alpha(f_k-f)(z)|:z\in K\}\le C\sup\{|f_k(z)-f(z)|:z\in\bar G\}
<C\varepsilon.\]
这正好说明$\DD^\alpha f_k$在$\Omega$上内闭一致收敛到$\DD^\alpha f$.
\end{proof}

多圆柱的Cauchy积分公式是单位圆盘Cauchy积分公式的自然推广,人们很早就知道它了.可是长期以来人们不知道球的Cauchy积分公式是什么样子,直到19世纪50年代中期,华罗庚得到了四类典型域的Cauchy积分公式,作为第一类典型域的一种特殊情形,人们才得到了球的Cauchy积分公式.由此可见,在多复变中,Cauchy积分公式因域而异,寻找给定域的Cauchy积分公式本身是一个相当困难的研究课题.
\begin{xiti}
\item 设$P(a,r)$是一个多圆柱,试利用多圆柱上的Cauchy积分公式证明:若$f\in C\big(\bar{P(a,r)}\big)\cap H\big(P(a,r)\big)$,则$f$的最大模必在$\partial_0P$上取到.
\item 设$\Omega$是$\MC^n$中的域,$\bar{P(a,r)}\subset\Omega$.如果$f\in H(\Omega)$,那么对任意多重指标$\alpha=(\alpha_1,\cdots,\alpha_n)$,有
    \[|(\DD^\alpha f)(a)|\le\frac{\alpha!(\alpha_1+2)\cdots(\alpha_n+2)}{(2\pi)^nr^{\alpha+2}}\|f\|
    _{L^1(P(a,r))},\]
这里
\[\|f\|_{L^1(P(a,r))}=\int\limits_{P(a,r)}|f(z)|\dif m(z),\]
$\dif m(z)$是$\MR^{2n}$中的测度.
\item 设$f$是$\MC^n$上的有界全纯函数,证明$f$必为常数.
\item 设$\Omega$是$\MC^n$中的域,$\{f_k\}$是$\Omega$上一列处处不为零的全纯函数.如果$\{f_k\}$在$\Omega$上内闭一致收敛于$f$,那么$f$在$\Omega$中或者恒等于零,或者处处不等于零.
\end{xiti}

\section{全纯函数在Reinhardt域上的展开式\label{sec9.3}}
为了找出发生Hartogs现象的域,我们要先讨论全纯函数在Reinhardt 域上的展开式.
\begin{definition}\label{def9.3.1}
设$\Omega$是$\MC^n$中的域,如果对任意$(z_1,\cdots,z_n)\in\Omega$及$\theta_1,\cdots,\theta_n\in\MR$,必有$(\ee^{\ii\theta_1}z_1,\cdots,\ee^{\ii\theta_n}z_n)\in\Omega$,就称$\Omega$是\textbf{Reinhardt域}\index{Y!域!Reinhardt域}.
\end{definition}

前面提到的球和多圆柱都是Reinhardt域.
\begin{theorem}\label{thm9.3.2}
设$\Omega$是$\MC^n$中的Reinhardt域,$f\in H(\Omega)$,那么$f$必有Laurent展开式
\begin{equation}\label{eq9.3.1}
f(z)=\sum_{\alpha\in\MZ^n}a_\alpha z^\alpha,
\end{equation}
这里,$\MZ^n=\{(\alpha_1,\cdots,\alpha_n):\alpha_1,\cdots,\alpha_n\text{都是整数}\}$.上述级数在$\Omega$中内闭一致收敛,且$a_\alpha$由$f$唯一确定.
\end{theorem}
\begin{proof}
我们先证,如果展开式 \eqref{eq9.3.1} 在$\Omega$上内闭一致收敛,那么$a_\alpha$
由$f$唯一确定.事实上,取$w\in\Omega$,要求$w$的每个坐标$w_j\ne0$.对于这个固定的$w$,取$z=(\ee^{\ii\theta_1}w_1,\cdots,\ee^{\ii\theta_n}w_n)$,则$z\in\Omega$.于是,由展开式 \eqref{eq9.3.1} 得
\[f(\ee^{\ii\theta_1}w_1,\cdots,\ee^{\ii\theta_n}w_n)=\sum_{\alpha\in\MZ^n}
a_\alpha w_1^{\alpha_1}\cdots w_n^{\alpha_n}\ee^{\ii(\alpha_1\theta_1+\cdots+\alpha_n\theta_n)}.\]
两端乘以$\ee^{-\ii(\beta_1\theta_1+\cdots+\beta_n\theta_n)}$,得
\[f(\ee^{\ii\theta_1}w_1,\cdots,\ee^{\ii\theta_n}w_n)
\ee^{-\ii(\beta_1\theta_1+\cdots+\beta_n\theta_n)}
=\sum_{\alpha\in\MZ^n}a_\alpha w_1^{\alpha_1}\cdots w_n^{\alpha_n}\ee^{\ii[(\alpha_1-\beta_1)\theta_1+\cdots+(\alpha_n-\beta_n)\theta_n]},\]
这里,$(\beta_1,\cdots,\beta_n)$是任意一个多重指标.上式两端分别对$\theta_1,\cdots,\theta_n$
在$[0,2\pi]$上积分,由于右端级数的项只有当$\alpha=\beta$时不为零,因而有
\[a_\beta=\frac1{(2\pi)^n}\frac1{w^\beta}\int_0^{2\pi}\cdots\int_0^{2\pi}
f(\ee^{\ii\theta_1}w_1,\cdots,\ee^{\ii\theta_n}w_n)
\ee^{-\ii(\beta_1\theta_1+\cdots+\beta_n\theta_n)}
\dif \theta_1\cdots\dif \theta_n.
\]
这就证明了展开式 \eqref{eq9.3.1} 中的系数由$f$所唯一确定.

现在证明展开式 \eqref{eq9.3.1} 成立.取定$w\in\Omega$,由于$\Omega$是Reinhardt域,故可取充分小的$\varepsilon>0$,使得
\[G(w,\varepsilon)=\{z\in\MC^n:|w_j|-\varepsilon<|z_j|<|w_j|+\varepsilon,j=1,\cdots,n\}\]
含在$\Omega$中.这是因为对于任意的$z\in G(w,\varepsilon)$,有
\[\big||z_j|-|w_j|\big|<\varepsilon,\;j=1,\cdots,n.\]
令$z_j'=\ee^{\ii\theta_j}z_j$,当然有
\begin{equation}\label{eq9.3.2}
\big||z_j'|-|w_j|\big|<\varepsilon,\;j=1,\cdots,n.
\end{equation}
适当选择$\theta_j$,可使$\arg z_j'=\arg w_j$,于是 \eqref{eq9.3.2} 式变成
\[|z_j'-w_j|<\varepsilon,j=1,\cdots,n.\]
因为$\Omega$是域,取$\varepsilon>0$充分小,可使$(z_1',\cdots,z_n')\in\Omega$. 因为$\Omega$是Reinhardt域,所以
\[(z_1,\cdots,z_n)=(\ee^{-\ii\theta_1}z_1',\cdots,\ee^{-\ii\theta_n}z_n')\in\Omega.\]
由于$G(w,\varepsilon)$是$n$个圆环的拓扑积,对$f$的每个变量分别用单复变中的Laurent定理,可得
\[f(z)=\sum_{\alpha\in\MZ^n}a_\alpha(w)z^\alpha,\;z\in G(w,\varepsilon),\]
它在$w$的邻域中一致收敛.不难证明$a_\alpha(w)$实际上与$w$无关.为此,取$w'\in G(w,\varepsilon)$,同样有
\[f(z)=\sum_{\alpha\in\MZ^n}a_\alpha(w')z^\alpha,\;z\in G(w',\varepsilon').\]
由上面证明的$a_\alpha$的唯一性知道,$a_\alpha(w)=a_\alpha(w'),\alpha\in\MZ^n $.这就证明了$a_\alpha(w)$在一个局部范围内是一个常数,利用$\Omega$的连通性,便知$a_\alpha(w)=a_\alpha$在$\Omega$上成立.因而
\[f(z)=\sum_{\alpha\in\MZ^n}a_\alpha z^\alpha\]
在$\Omega$中每一点的邻域中一致地成立,从而在$\Omega$中内闭一致地成立.
\end{proof}

从这个定理可得下面很有用的定理:
\begin{theorem}\label{thm9.3.3}
设$\Omega$是$\MC^n$中的Reinhardt域,如果对每个$j$
($j=1,\cdots,n$),$\Omega$中都有$j$第个坐标为零的点,那么每个$f\in H(\Omega)$都有幂级数展开式
\begin{equation}\label{eq9.3.3}
f(z)=\sum_{\alpha\ge0}a_\alpha z^\alpha,
\end{equation}
它在$\Omega$上内闭一致地成立.
\end{theorem}
\begin{proof}
因为$\Omega$是Reinhardt域,根据定理 \ref{thm9.3.2},$f$有Laurent展开式
\begin{equation}\label{eq9.3.4}
f(z)=\sum_{\alpha\in\MZ^n}a_\alpha z^\alpha.
\end{equation}
我们要证明,如果$\alpha=(\alpha_1,\cdots,\alpha_n)$的某个分量出现负整数,那么相应的系数$a_\alpha=0$,这样展开式 \eqref{eq9.3.4} 就变成幂级数了.事实上,设$\alpha_k$是一个负整数,而相应的$a_\alpha\ne0$,这时取第$k$个坐标为零的点
\[\tilde{z}=(z_1,\cdots,z_{k-1},0,z_{k+1},\cdots,z_n),\]
于是$a_\alpha\tilde{z}^\alpha=\infty$,因而 \eqref{eq9.3.4} 式在$\tilde{z}$处不成立,这是一个矛盾. 所以,\eqref{eq9.3.4} 式的右端就是幂级数 \eqref{eq9.3.3}.
\end{proof}

由于单位球$B_n$和单位多圆柱$U^n$都满足定理 \ref{thm9.3.3} 的条件,所以从定理 \ref{thm9.3.3} 立刻可得
\begin{theorem}\label{thm9.3.4}
每个$f\in H(B_n)$都有幂级数展开式
\[f(z)=\sum_{\alpha\ge0}a_\alpha z^\alpha,\;z\in B_n.\]
\end{theorem}

\begin{theorem}\label{thm9.3.5}
每个$f\in H(U^n)$都有幂级数展开式
\[f(z)=\sum_{\alpha\ge0}a_\alpha z^\alpha,\;z\in U^n.\]
\end{theorem}

上面这个展开式也可写为
\[f(z)=\sum_{k=0}^\infty\sum_{|\alpha|=k}a_\alpha z^\alpha=\sum_{k=0}^\infty P_k(z),\]
这里,$P_k(z)=\sum_{|\alpha|=k}a_\alpha z^\alpha$是$z_1,\cdots,z_n$的$k$次齐次多项式.上式称为$f$的\textbf{齐次展开式}\index{Q!齐次展开式}.

为简单起见,我们就$n=2$的情形写出$P_k(z)$的具体表达式:
\begin{align*}
&P_0(z)=a_{0,0},\\
&P_1(z)=\sum_{|\alpha|=1}a_\alpha z^\alpha=a_{1,0}z_1+a_{0,1}z_2,\\
&P_2(z)=\sum_{|\alpha|=2}a_\alpha z^\alpha=a_{2,0}z_1^2+a_{1,1}z_1z_2+a_{0,2}z_2^2,\\
&P_3(z)=\sum_{|\alpha|=3}a_\alpha z^\alpha=a_{3,0}z_1^3+a_{2,1}z_1^2z_2+a_{1,2}z_1z_2^2+a_{0,3}z_2^3,\\
&\cdots.
\end{align*}

从定理 \ref{thm9.3.3} 还可得到下面的\textbf{全纯开拓定理}\index{D!定理!全纯开拓定理}:
\begin{theorem}\label{thm9.3.6}
设$\Omega$是$\MC^n$中的Reinhardt域,如果对每个$j$($j=1,\cdots,n$),$\Omega$中都有第$j$个坐标为零的点,那么每个$f\in H(\Omega)$都能全纯开拓到域
\[\Omega'=\{w=(\rho_1z_1,\cdots,\rho_nz_n):(z_1,\cdots,z_n)\in\Omega,0\le\rho_j\le1,
j=1,\cdots,n\}.\]
\end{theorem}
\begin{proof}
根据定理 \ref{thm9.3.3},$f$在$\Omega$中有幂级数展开式
\[f(z)=\sum_{\alpha\ge0}a_\alpha z^\alpha,\;z\in\Omega.\]
任取$w\in\Omega'$,按定义,存在$z\in\Omega$及$0\le\rho_j\le1,j=1,\cdots,n$,使得$w_j=\rho_jz_j,j=1,\cdots,n$,因而$|w_j|\le|z_j|$. 由于$\sum_{\alpha\ge 0}a_\alpha z^\alpha$收敛,由命题 \ref{prop9.1.1},$\sum_{\alpha\ge0}a_\alpha w^\alpha$收敛,且在$\Omega'$中内闭一致收敛. 现在定义
\[F(w)=\sum_{\alpha}a_\alpha w^\alpha,\;w\in\Omega',\]
由Weierstrass定理,$F\in H(\Omega')$,且$F\big|_\Omega=f$,所以$F$是$f$在$\Omega'$上的全纯开拓.
\end{proof}

从这个定理马上可以举出发生Hartogs现象的具体的域.
\begin{example}\label{exam9.3.7}
设$0<r<R$,若
\[\Omega=\{z=(z_1,\cdots,z_n):r^2<|z_1^2|+\cdots+|z_n|^2<R^2\},\]
则每个$f\in H(\Omega)$必能全纯开拓到
\[B(0,R)=\{z=(z_1,\cdots,z_n):|z_1|^2+\cdots+|z_n|^2<R^2\}.\]
\end{example}

由例 \ref{exam9.3.7} 还可得到一个与单复变有本质不同的事实:在单复变中,全纯函数的零点一定是孤立的,可在多复变中恰好相反.
\begin{theorem}\label{thm9.3.8}
  设$\Omega$是$\MC^n$($n>1$)中的域,$f\in H(\Omega)$,那么$f$在$\Omega$上的零点一定不是孤立的.
\end{theorem}
\begin{proof}
如果$a\in\Omega$是$f$的一个孤立零点,这意味着存在以$a$为中心、以充分小的正数$\varepsilon$为半径的球$B(a,\varepsilon)\subset\Omega$,$f$在$B(a,\varepsilon)$中除$a$以外不再有其他的零点.令
\[g(z)=\frac1{f(z)},\]
则$g$在$B(a,\varepsilon)\backslash\bar{B\bigg(a,\frac\varepsilon2\bigg)}$中全纯. 由例 \ref{exam9.3.7},$g$必在$B(a,\varepsilon)$中全纯,因而$f(a)\ne0$,这是一个矛盾.
\end{proof}

Hartogs现象是多复变数空间$\MC^n$($n>1$)中所特有的,在复平面上没有这种现象.设$D$是$\MC$中的域,在$\partial D$上任意取一点$a$,那么
\[f(z)=\frac1{z-a}\]
是$D$中的全纯函数,但它不能越过$a$全纯开拓出去.这说明对$D$不会发生Hartogs现象。

$\MC^n$中怎样的域不会发生Hartogs现象呢?我们把不发生Hartogs现象的域称为全纯域.严格说来,我们有下面的
\begin{definition}\label{def9.3.9}
设$\Omega$是$\MC^n$中的域,如果不存在比$\Omega$更大的域$\Omega'$
($\Omega'\supset\Omega,\Omega'\ne\Omega$),使得$H(\Omega)$中的每个函数都能全纯开拓到$\Omega'$上去,就称$\Omega$是\textbf{全纯域}\index{Y!域!全纯域}.
\end{definition}

如上面所说,$\MC$中所有的域都是全纯域. $\MC^n$($n>1$)中什么样的域是全纯域,这是多复变中一个十分重要的问题.我们在这里给出一个域是全纯域的一个充分条件.
\begin{theorem}\label{thm9.3.10}
$\MC^n$中的凸域一定是全纯域.
\end{theorem}
\begin{proof}
设$\Omega$是$\MC^n$中的凸域,故对$\partial\Omega$上每一点$\zeta$,存在一个过$\zeta$的超平面$Q$,它与$\Omega$不相交.不妨设$\zeta=0$,则过$\zeta$的超平面可写为
\begin{equation}\label{eq9.3.5}
a_1x_1+b_1y_1+\cdots+a_nx_n+b_ny_n=0.
\end{equation}
记$c_j=a_j-\ii b_j,z_j=x_j+\ii y_j$,则 \eqref{eq9.3.5} 式可写为
\[\Re\bigg(\sum_{j=1}^nc_jz_j\bigg)=0.\]
令$g(z)=\sum_{j=1}^nc_jz_j$,易知$g$在$\Omega$中没有零点. 因为若有$w\in\Omega$,使得$g(w)=0$,则$\Re g(w)=0$,这等于说超平面 \eqref{eq9.3.5} 通过$w$点,与$Q$的取法矛盾. 因而
\[f(z)=\frac1{g(z)}\]
是$\Omega$中的全纯函数,但它不能通过边界点$\zeta$全纯开拓出去.
\end{proof}

显然,这个定理的逆是不成立的,即全纯域不一定是凸域,这在$n=1$时就有大量的反例.因此,全纯域是比凸域更为广泛的一类域.
\begin{xiti}
\item 记$A(B_n)=H(B_n)\cap C(\bar{B_n}),n>1$.
\begin{enuma}
\item 设$f\in A(B_n)$,如果$f$在$\partial B_n$上处处不为零,证明$f$在$B_n$中也处处不为零;
\item 设$f\in A(B_n)$,证明$f(\bar{B_n})=f(\partial B_n)$;
\item 举例说明,上述结论在$n=1$时不成立.
\end{enuma}
\item 设$f\in A(B_n),n>1$.如果$|f|$在球面$\partial B_n$上恒等于常数$c$,那么$f$在$B_n$中也恒等于常数$c$.举例说明$n=1$时结论不成立.
\item 设$0<\alpha,\beta<1$,记
\begin{align*}
&G_1=\{(z,w)\in\MC^2:|z|<1,\beta<|w|<1\},\\
&G_2=\{(z,w)\in\MC^2:|z|<\alpha,|w|<1\},
\end{align*}
令$G=G_1\cup G_2$. 证明:$H(G)$中的每一个函数都能全纯开拓到双圆柱域
\[U^2=\{(z,w)\in\MC^2:|z|<1,|w|<1\}.\]
\end{xiti}

\section{全纯映射的导数\label{sec9.4}}
设$\Omega$是$\MC^n$中的域,$f:\Omega\to \MC$是定义在$\Omega$上的复值函数,它可以看成是$\Omega$到$\MC$上的一个映射.从映射的角度来看,更重要的是要考虑$\Omega$到$\MC^n$中的映射.
\begin{theorem}\label{def9.4.1}
设$\Omega$是$\MC^n$中的域,$f_1,\cdots,f_n$是$\Omega$上的$n$个全纯函数,称$F=(f_1,\cdots,f_n)$是$\Omega$到$\MC^n$的\textbf{全纯映射}\index{Q!全纯映射}.
\end{theorem}

在单变数的情形下,如果$f$在$z_0$处可微,则有
\[f(z_0+h)-f(z_0)=f'(z_0)h+o(|h|),\]
这里,导数$f'(z_0)$可以看成是由$f$和$z_0$确定的一个线性算子$A:h\to f'(z_0)h$.下面就用这一观点来定义全纯映射的导数.

\begin{definition}\label{def9.4.2}
设$\Omega$是$\MC^n$中的域,$F:\Omega\to\MC^n$是一个映射. 对于给定的$z\in\Omega$,如果存在$A\in L(\MC^n,\MC^n)$,使得
\begin{equation}\label{eq9.4.1}
F(z+h)-F(z)=Ah+o(|h|),
\end{equation}
就称$F$在$z$点\textbf{可微}\index{Q!全纯映射!可微},称$A$为$F$在$z$点的\textbf{导数}\index{Q!全纯映射!导数},记为$F'(z)=A$.这里,$h\in \MC^n,L(\MC^n,\MC^n)$表示$\MC^n\to\MC^n$的线性映射的全体,\eqref{eq9.4.1} 式的含义是
\[\lim_{h\to0}\frac{|F(z+h)-F(z)-Ah|}{|h|}=0.\]
\end{definition}

那么什么样的映射可微呢?我们有下面的
\begin{theorem}\label{thm9.4.3}
设$\Omega$是$\MC^n$中的域,$F=(f_1,\cdots,f_n)$是$\Omega$上的全纯映射,那么$F$在$\Omega$中的每一点都可微,而且对任意$z\in\Omega$,有
\begin{equation}\label{eq9.4.2}
F'(z)=\begin{pmatrix}
\pp{f_1(z)}{z_1}&\cdots&\pp{f_1(z)}{z_n}\\
\cdots&\cdots&\cdots\\
\pp{f_n(z)}{z_1}&\cdots&\pp{f_n(z)}{z_n}
\end{pmatrix}.
\end{equation}
\end{theorem}
\begin{proof}
因为每个$f_j$都是$\Omega$上的全纯函数,故在$z$的邻域中可以展开为幂级数:
\begin{align*}
&f_1(z+h)=f_1(z)+\pp{f_1(z)}{z_1}h_1+\cdots+\pp{f_1(z)}{z_n}h_n+o(|h|),\\
&\cdots,\\
&f_n(z+h)=f_n(z)+\pp{f_n(z)}{z_1}h_1+\cdots+\pp{f_n(z)}{z_n}h_n+o(|h|).
\end{align*}
写成向量的形式,就有
\[F(z+h)=F(z)+Ah+o(|h|),\]
这里,$A$就是 \eqref{eq9.4.2} 式右边的方阵. 按照定义 \ref{def9.4.2},$F$在$z$点可微,而且$F'(z)=A$.
\end{proof}

由此可知,全纯映射$F$的导数就是它的\textbf{Jacobian矩阵}\index{J!Jacobian矩阵}.

下面我们将证明,两个全纯映射的复合映射也是全纯的,而且有类似于复合函数求导数的求导法则.
\begin{prop}\label{prop9.4.4}
设$\Omega_1,\Omega_2$是$\MC^n$中的两个域,如果
\begin{align*}
&F:\Omega_1\to\Omega_2,\\
&G:\Omega_2\to\MC^n
\end{align*}
都是全纯映射,那么复合映射$H=G\circ F$也是$\Omega_1$上的全纯映射,而且
\[H'(z)=G'(w)F'(z),\]
其中,$w=F(z)$.
\end{prop}
\begin{proof}
设$F=(f_1,\cdots,f_n),G=(g_1,\cdots,g_n),H=(h_1,\cdots,h_n)$,其中
\[h_j=g_j(f_1,\cdots,f_n),\;j=1,\cdots,n.\]
于是
\begin{align}
&\pp{h_j}{\bar{z}_l}=\sum_{s=1}^n\bigg(\pp{g_j}{w_s}\pp{w_s}{\bar{z}_l}
+\pp{g_j}{\bar{w}_s}\pp{\bar{w}_s}{\bar{z}_l}\bigg),\label{eq9.4.3}\\
&\pp{h_j}{z_l}=\sum_{s=1}^n\bigg(\pp{g_j}{w_s}\pp{w_s}{z_l}
+\pp{g_j}{\bar{w}_s}\pp{\bar{w}_s}{z_l}\bigg).\label{eq9.4.4}
\end{align}
由Cauchy--Riemann方程组,有
\begin{align*}
&\pp{w_s}{\bar{z}_l}=0,s=1,\cdots,n\\
&\pp{g_j}{\bar{w}_s}=0,s=1,\cdots,n.
\end{align*}
\eqref{eq9.4.3} 式和 \eqref{eq9.4.4} 式分别变为
\begin{align}
&\pp{h_j}{\bar{z}_l}=0,\;j,l=1,\cdots,n,\label{eq9.4.5}\\
&\pp{h_j}{z_l}=\sum_{s=1}^n\pp{g_j}{w_s}\pp{w_s}{z_l},\;j,l=1,\cdots,n.\label{eq9.4.6}
\end{align}
从 \eqref{eq9.4.5} 式即得$h\in H(\Omega_1),j=1,\cdots,n$,所以$H$是全纯映射.由 \eqref{eq9.4.6} 式即得
\begin{equation*}
  H'(z) = G'(w)F'(z). \qedhere
\end{equation*}
\end{proof}

\begin{xiti}\hypertarget{xiti9.4}{}
\item \hypertarget{xiti9.4.1}{} 设$f$是$\MC$上的一个整函数,满足$f(0)=f'(0)=0$.令$F(z_1,z_2)=\big(z_1,z_2+f(z_1)\big), I_2$为$2$阶单位方阵.证明:
    \[F(0)=0,F'(0)= I_2.\]
\item \hypertarget{xiti9.4.2}{} 设$\Omega=\{z=(z_1,z_2)\in\MC^2:|z_1z_2|<1\}$,任取$h:U\to\MC$为$U$中处处不为零的全纯函数,这里,$U$表示单位圆盘. 令
    \[F_h(z_1,z_2)=\bigg(z_1h(z_1z_2),\frac{z_2}{h(z_1z_2)}\bigg).\]
证明:
\begin{enuma}
   \item $F_h(0)=0$;
   \item $(F_h)^{-1}=F_{\frac1h}$;
   \item 如果$h(0)=1$,那么$F_h'(0)= I_2$.
\end{enuma}
\item 设$\Omega$是$\MC^n$中的域,$F:\Omega\to\MC^n$是全纯映射. 设$F=(f_1,\cdots,f_n),f_j=u_j+\ii v_j,j=1,\cdots,n$. 记
    \begin{align*}
    &(\textrm JF)(z)=\det F'(z),\\
    &(\textrm J_\MR F)(z)=\det\begin{pmatrix}
    \pp{(u_1,\cdots,u_n)}{(x_1,\cdots,x_n)}&\pp{(u_1,\cdots,u_n)}{(y_1,\cdots,y_n)}\\
    \pp{(v_1,\cdots,v_n)}{(x_1,\cdots,x_n)}&\pp{(v_1,\cdots,v_n)}{(y_1,\cdots,y_n)}
    \end{pmatrix},
    \end{align*}
前者称为$F$的\textbf{复Jacobian}\index{J!Jacobian矩阵!复Jacobian},后者称为$F$的\textbf{实Jacobian}\index{J!Jacobian矩阵!实Jacobian}.证明:
\[(\textrm J_\MR F)(z)=|(\textrm JF)(z)|^2.\]
\item 设$\Omega$是$\MC^n$中的域,$F:\Omega\to \MC^n$是全纯映射.如果存在$a\in\Omega$,使得$F'(a)$可逆,那么一定存在$a$和$F(a)$的邻域$V$和$W$,使得$F$一一地把$V$映为$W$,而且$F$的逆映射$G:W\to V$是全纯的,等式
    \[G'(w)=\big(F'(z)\big)^{-1}\]
对$z\in V$成立.
\end{xiti}

\section{Cartan定理\label{sec9.5}}
定理 \ref{thm4.5.5} 给出了单位圆盘的全纯自同构.一个自然的问题是,如何确定$\MC^n$中单位球的全纯自同构?在证明定理 \ref{thm4.5.5} 时,用到了Schwarz引理(定理 \ref{thm4.5.3})的第 \ref{thm4.5.3.3} 个结论,即若$f$是单位圆盘到单位圆盘的映射,$f(0)=0$,如果$f'(0)=1$,那么$f(z)=z$. H. Cartan把这个定理推广到了多复变数.

\begin{theorem}\label{thm9.5.1}\index{D!定理!Cartan定理}
设$\Omega$是$\MC^n$中包含原点的有界域,如果$F:\Omega\to\Omega$是全纯的,且有$F(0)=0,F'(0)= I_n$,这里,$I_n$是$n$阶单位方阵,那么对任意$z\in\Omega$,有$F(z)=z$.
\end{theorem}
\begin{proof}
因为$\Omega$是有界域,故必存在$0<r<R<\infty$,使得$B(0,r)\subset\Omega\subset B(0,R)$. 设$F=(f_1,\cdots,f_n)$,由假定得
\[f_j(0)=0,\;\pp{f_i}{z_j}(0)=\delta_{ij},\;i,j=1,\cdots,n.\]
根据定理 \ref{thm9.3.4},$f_j$在$B(0,r)$中有展开式
\[f_j(z)=z_j+P_2^{(j)}(z)+\cdots,j=1,\cdots,n,\]
这里,$P_2^{(j)},P_3^{(j)},\cdots$分别是$z_1,\cdots,z_n$的$2$次、$3$次、$\cdots$齐次多项式. 因而上式可写为
\begin{equation}\label{eq9.5.1}
F(z)=z+\sum_{s=2}^\infty F_s(z),
\end{equation}
这里,$F_s(z)=\big(P_s^{(1)}(z),\cdots,P_s^{(n)}(z)\big)$. 我们要证明
\[F_j(z)\equiv0,\;s=2,3,\cdots.\]
设第一个不为零的是$F_m,m\ge2$,则 \eqref{eq9.4.1} 式可写为
\begin{equation}\label{eq9.5.2}
F(z)=z+\sum_{s=m}^\infty F_s(z).
\end{equation}
注意到$P_m^{(j)}(z)=\sum_{|\alpha|=m}a_\alpha z^\alpha$,所以
\begin{align*}
P_m^{(j)}\big(F(z)\big)&=P_m^{(j)}(f_1,\cdots,f_n)=\sum_{|\alpha|=m}a_\alpha
f_1^{\alpha_1}\cdots f_n^{\alpha_n}\\
&=\sum_{|\alpha|=m}a_\alpha(z_1+\cdots)^{\alpha_1}\cdots(z_n+\cdots)^{\alpha_n}\\
&=\sum_{|\alpha|=m}a_\alpha z^\alpha+\cdots=P_m^{(j)}(z)+\cdots,
\end{align*}
因而
\[F_m\big(F(z)\big)=F_m(z)+\cdots.\]
记$F^2=F\circ F,F^k=F\circ F^{k-1}$,那么
\begin{align*}
F^2(z)&=F\big(F(z)\big)\\
&=F(z)=F_m\big(F(z)\big)+\cdots\\
&=z+2F_m(z)+\cdots.
\end{align*}
一般可得
\[F^k(z)=z+kF_m(z)+\cdots,\;k=1,2,\cdots,\;z\in B(0,r).\]
因为$F_m(z)$的每个分量都是$m$次齐次多项式,所以
\[F^k(\ee^{\ii\theta}z)=\ee^{\ii\theta}z+k\ee^{\ii m\theta}F_m(z)+\cdots.\]
两端乘以$\ee^{-\ii m\theta}$,并对$\theta$从$0$到$2\pi$积分,得
\begin{equation}\label{eq9.5.3}
kF_m(z)=\frac1{2\pi}\int_0^{2\pi}F^k(\ee^{\ii\theta}z)\ee^{-\ii m\theta}\dif \theta,
\end{equation}
这里,积分是对$F$的每个分量进行的.因为$F$把$\Omega$映入$\Omega$,所以$F^k$也把$\Omega$映入$\Omega$,因而
\[|F^k(\ee^{\ii\theta}z)|<R,\;k=1,2,\cdots\]
对所有$z\in B(0,r)$及$0\le\theta\le2\pi$成立. 由 \eqref{eq9.5.3} 式即得
\[k|F_m(z)|<R,\;k=1,2,\cdots\]
对每个$z\in B(0,r)$成立,因而在$B(0,r)$上有$F_m(z)\equiv0$. 再由唯一性定理(定理 \ref{thm9.1.4}),$F_m(z)\equiv0$在$\Omega$上成立,所以在$\Omega$上有$F(z)\equiv z$.
\end{proof}

Cartan证明的另一个定理是对圆型域上的双全纯映射来说的.
\begin{definition}\label{def9.5.2}
设$\Omega$是$\MC^n$中的域,如果对任意$z\in\Omega$及实数$\theta$,均有$\ee^{\ii\theta}z\in\Omega$,就称$\Omega$为\textbf{圆型域}\index{Y!域!圆型域}.
\end{definition}

显然,球和多圆柱都是圆型域,Reinhardt域也一定是圆型域,但圆型域不一定是Reinhardt域.例如,$\Omega=\{(z_1,\cdots,z_n):|z_1+\cdots+z_n|<1\}$显然是圆型域,但它不是Reinhardt域.
\begin{definition}\label{def9.5.3}
设$\Omega$是$\MC^n$中的域,$F:\Omega\to\MC^n$是全纯映射.如果$F$有全纯的逆映射$F^{-1}$,就称$F$是\textbf{双全纯映射}.\index{Q!全纯映射!双全纯映射}
\end{definition}

对于圆型域上的双全纯映射,有下面的
\begin{theorem}\label{thm9.5.4}
设$\Omega_1$和$\Omega_2$是$\MC^n$中包含原点的圆型域,其中$\Omega_1$是有界的.如果$F:\Omega_1\to\Omega_2$是双全纯映射,且$F(0)=0$,那么$F$一定是线性映射.
\end{theorem}
\begin{proof}
令$G=F^{-1}$,因为$G\big(F(z)\big)=z$,对等式两边求导数,由命题 \ref{prop9.4.4} 得$G'(0)F'(0)$\\$= I_n$. 对于固定的实数$\theta$,定义
\[H(z)=G\big(\ee^{-\ii\theta}F(\ee^{\ii\theta}z)\big),\;z\in\Omega_1.\]
因为$\Omega_1$和$\Omega_2$都是圆型域,所以$H$是$\Omega_1\to\Omega_1$的全纯映射,而且
$H(0)=0,H'(0)=G'(0)F'(0)= I_n$. 又因为$\Omega_1$是有界域,于是由定理 \ref{thm9.5.1},$H(z)=z$对$z\in\Omega_1$成立. 由此可得
\[F(z)=F\big(H(z)\big)=F\big[G\big(\ee^{-\ii\theta}F(\ee^{\ii\theta}z)\big)\big]
=\ee^{-\ii\theta}F(\ee^{\ii\theta}z),\]
即
\begin{equation}\label{eq9.5.4}
F(\ee^{\ii\theta}z)=\ee^{\ii\theta}F(z),
\end{equation}
对任意$z\in\Omega_1$及实数$\theta$成立.今在原点附近作$F(z)$的齐次展开式:
\begin{equation}\label{eq9.5.5}
F(z)=F'(0)z+F_m(z)+\cdots,\;m\ge2,
\end{equation}
这里,$F(z),z,F_m(z)$都写成列向量,$F_m(z)$的每个分量都是$m$次齐次多项式.在 \eqref{eq9.5.5} 式中用$\ee^{\ii\theta}z$代替$z$,并注意到等式 \eqref{eq9.5.4},即得
\begin{equation}\label{eq9.5.6}
F(z)=F'(0)z+\ee^{\ii(m-1)\theta}F_m(z)+\cdots.
\end{equation}
比较 \eqref{eq9.5.5} 式与 \eqref{eq9.5.6} 式,即得$F_m(z)=0$,因而
\[F(z)=F'(0)z.\]
这就证明了$F$是一个线性映射.
\end{proof}

设$\Omega$是$\MC^n$中的域,如果$F$是把$\Omega$映为自己的双全纯映射,就称$F$是$\Omega$的一个\textbf{全纯自同构}\index{Q!全纯自同构}.与单复变的情形一样,$\Omega$的全纯自同构的全体记为$\Aut(\Omega)$.
\begin{definition}\label{def9.5.5}
设$\Omega$是$\MC^n$中的域,如果对$\Omega$中任意两点$a,b$,必有$\varphi\in\Aut(\Omega)$,使得$\varphi(a)=b$,就称$\Omega$为\textbf{可递域}\index{Y!域!可递域}或\textbf{齐性域}.
\index{Y!域!齐性域}
\end{definition}

可递域有很多良好的性质.
\begin{theorem}\label{thm9.5.6}
设$\Omega_1$和$\Omega_2$是$\MC^n$中包含原点的圆型域,其中$\Omega_1$是有界的可递域. 如果存在双全纯映射$F$把$\Omega_1$映为$\Omega_2$,那么一定存在线性映射把$\Omega_1$映为$\Omega_2$.
\end{theorem}
\begin{proof}
设$a=F^{-1}(0)$. 因为$\Omega_1$是可递的,故必存在$\varphi\in\Aut(\Omega_1)$,使得$\varphi(0)=a$. 令$G=F\circ \varphi$,于是
\[G(\Omega_1)=F\big(\varphi(\Omega_1)\big)=F(\Omega_1)=\Omega_2,\]
而且$G(0)=F\big(\varphi(0)\big)=F(a)=0$. 应用定理 \ref{thm9.5.4},即知$G$是线性映射.
\end{proof}

\begin{xiti}
\item 由习题  \hyperlink{xiti9.4}{9.4} 中的第 \hyperlink{xiti9.4.1}{1},\hyperlink{xiti9.4.2}{2} 两题说明,定理 \ref{thm9.5.1} 和定理 \ref{thm9.5.4} 中域$\Omega$和$\Omega_1$有界的条件是不能去掉的.
\item 设$F=(f_1,f_2):U^2\to B_2$是双全纯映射.
\begin{enuma}
  \item 证明:对于任意$\{z_k\}\subset U$($U$是单位圆盘),$|z_k|\to1$,必有子列$\{z_{k_v}\}$,使得
      \[\lim_{v\to\infty}F(z_{k_v},w)=\varphi(w),\]
  其中,$\varphi$是常数映射,即$\varphi(w)=(c_1,c_2)$;
  \item 由此证明$F$不存在,即不存在把$U^2$一一地映为$B_2$的双全纯映射.
\end{enuma}
\item 设$f\in\Aut(U^n),U^n$是单位多圆柱.
\begin{enuma}
  \item 如果$f(0)=0$,那么一定存在实数$\theta_1,\cdots,\theta_n$和置换$\tau:(1,\cdots,n)\to(1,\cdots,n)$,使得
      \[f(z)=\big(\ee^{\ii\theta_1}z_{\tau(1)},\cdots,\ee^{\ii\theta_n}z_{\tau(n)}\big);\]
  \item 如果$f(a)=0,a\ne0$,那么一定存在实数$\theta_1,\cdots,\theta_n$和置换$\tau:(1,\cdots,n)\to(1,\cdots,n)$,使得
  \[f(z)=\bigg(\ee^{\ii\theta_1}\frac{z_{\tau(1)}-a_1}{1-\bar{a_1}z_{\tau(1)}},
  \cdots,\ee^{\ii\theta_n}\frac{z_{\tau(n)}-a_n}{1-\bar{a_n}z_{\tau(n)}}\bigg).\]
\end{enuma}
\end{xiti}

\section{球的全纯自同构和Poincar\'e定理\label{sec9.6}}
在这一节中,我们要定出单位球的全部自同构.

先定出让原点保持不变的全纯自同构.
\begin{theorem}\label{thm9.6.1}
设$\varphi\in\Aut(B_n)$,如果$\varphi(0)=0$,那么$\varphi$是一个酉变换,即存在酉方阵$U$,使得
\[\varphi(z)=zU,\;z\in B_n.\]
\end{theorem}
\begin{proof}
由定理 \ref{thm9.5.4} 知道,$\varphi$是一个线性变换.设$\varphi(z)=zT$,这里,$T$是一个$n$阶可逆方阵. 对于任意$z\in B_n$,如果$|z|<|zT|$,那么$\frac z{|zT|}\in B_n$,因而$\bigg(\frac z{|zT|}\bigg)T\in B_n$,但$\bigg|\bigg(\frac z{|zT|}\bigg)T\bigg|=1$,这不可能. 同理可证,$|z|>|zT|$也别可能. 因而对所有$z\in B_n$,都有$|z|=|zT|$,这说明$T$是酉方阵.
\end{proof}

在下面的讨论中,$z=(z_1,\cdots,z_n),a=(a_1,\cdots,a_n)$都表示行向量,$z',a'$表示它们的转置,是一个列向量,因而$az'$表示一个数,而$a'z$表示一个方阵:
\begin{align*}
&az'=(a_1,\cdots,a_n)\begin{pmatrix}
z_1\\\vdots\\z_n
\end{pmatrix}=a_1z_1+\cdots+a_nz_n,\\
&a'z=\begin{pmatrix}
a_1\\\vdots\\a_n
\end{pmatrix}(z_1,\cdots,z_n)=\begin{pmatrix}
a_1z_1&\cdots&a_1z_n\\
\cdots&\cdots&\cdots\\
a_nz_1&\cdots&a_nz_n
\end{pmatrix}.
\end{align*}

\begin{theorem}\label{thm9.6.2}
对于每个$a\in B_n$,记$s^2=1-|a|^2,A=sI_n+\frac{\bar {a'}a}{1+s}$,那么映射
\[\varphi_a(z)=\frac{a-zA}{1-\bar az'}\]
具有下列性质:
\begin{eenum}
 \item \label{thm9.6.2.1} $\varphi_a(0)=a,\varphi_a(a)=0$;
 \item \label{thm9.6.2.2} $\varphi_a'(0)=a'\bar a-A',\varphi_a'(a)=-\frac{A'}{s^2};$
 \item \label{thm9.6.2.3} 对$z\in\bar{B_n}$,有
 \[1-|\varphi_a(z)|^2=\frac{\big(1-|a|^2\big)\big(1-|z|^2\big)}{|1-\bar az'|^2};\]
 \item \label{thm9.6.2.4} $\varphi_a\big(\varphi_a(z)\big)=z$;
 \item \label{thm9.6.2.5} $\varphi_a\in\Aut(B_n)$.
\end{eenum}
\end{theorem}
\begin{proof}
(1) $\varphi_a(0)=0$是显然的. 由于
\[aA=sa+\frac{a\bar {a'}a}{1+s}=sa+(1-s)a=a,\]
所以$\varphi_a(a)=0$.

(2)因为
\begin{align*}
\varphi_a(z)&=\big(a-zA)(1+\bar az'+o(|z|)\big)\\
&=a-zA+z\bar {a'}a+o(|z|)\\
&=\varphi_a(0)+z(\bar {a'}a-A)+o(|z|),
\end{align*}
根据全纯映射导数的定义(定义 \ref{def9.4.2}),即得
\[\varphi_a'(0)=a'\bar a-A'.\]
为了计算$\varphi_a'(a)$,注意
\begin{align*}
\varphi_a(a+h)-\varphi_a(a)&=\varphi_a(a+h)=\frac{a-(a+h)A}{1-\bar a(a+h)'}\\
&=-\frac{hA}{s^2}\bigg(1-\frac{\bar ah'}{s^2}\bigg)^{-1}=
-\frac{hA}{s^2}\bigg(1+\frac{\bar ah'}{s^2}+o(|h|)\bigg)\\
&=-\frac{hA}{s^2}+o(|h|),
\end{align*}
由此即得
\[\varphi_a'(a)=-\frac{A'}{s^2}.\]

(3)通过直接计算,得
\begin{equation}\label{eq9.6.1}
\begin{aligned}
1-|\varphi_a(z)|^2={}&1-\varphi_a(z)\bar{\varphi_a(z)'}\\
={}&1-\frac{a-zA}{1-\bar az'}\frac{\bar{a'}-\bar{A'}\bar{z'}}{1-a\bar{z'}}\\
={}&\frac1{|1-\bar az'|^2}(1-\bar az'-a\bar{z'}+\bar az'z\bar{z'}\\
&-a\bar{a'}+zA\bar{a'}+a\bar{A'}\bar{z'}-zA\bar{A'}\bar{z'}).
\end{aligned}
\end{equation}
注意到
\begin{align*}
&a\bar{A'}=aA=a,\\
&A\bar{a'}=\bar{a'},
\end{align*}
所以
\begin{align*}
&a\bar{A'}\bar{z'}=a\bar{z'},\\
&zA\bar{a'}=z\bar{a'}=\bar az'.
\end{align*}
显然$\bar{A'}=A$,所以
\begin{align*}
A\bar{A'}&=A^2=s^2I_n+\frac{2s\bar{a'}a}{1+s}+\frac{\bar{a'}a\bar{a'}a}{(1+s)^2}\\
&=s^2I_n+\frac{2s\bar{a'}a}{1+s}+\frac{1-s}{1+s}\bar{a'}a\\
&=s^2I_n+\bar{a'}a,
\end{align*}
即
\[\bar {a'}a-A\bar{A'}=-s^2I_n.\]
于是,\eqref{eq9.6.1} 式可写为
\begin{align*}
1-|\varphi_a(z)|^2&=\frac1{|1-\bar az'|^2}\big(1-|a|^2+z\bar{a'}a\bar{z'}-
zA\bar{A'}\bar{z'}\big)\\
&=\frac1{|1-\bar az'|^2}\big(1-|a|^2+z(\bar{a'}a-A\bar{A'})\bar{z'}\big)\\
&=\frac1{|1-\bar az'|^2}\big(1-|a|^2-s^2z\bar{z'}\big)\\
&=\frac1{|1-\bar az'|^2}\big(1-|a|^2\big)\big(1-|z|^2\big.)
\end{align*}
这就是要证明的.

(4) 记$H=\varphi_a\circ\varphi_a$. 由 \ref{thm9.6.2.3} 知$\varphi_a$是把$B_n$映入$B_n$的全纯映射,所以$H$也是把$B_n$映入$B_n$的全纯映射. 由 \ref{thm9.6.2.1} 得
\[H(0)=\varphi_a\big(\varphi_a(0)\big)=\varphi_a(a)=0.\]
再由命题 \ref{prop9.4.4} 及 \ref{thm9.6.2.2},可得
\[H'(0)=\varphi_a'(a)\varphi_a'(0)=-\frac{A'}{s^2}(a'\bar a-A')=I_n.\]
于是由定理 \ref{thm9.5.1},即得
\[\varphi_a\big(\varphi_a(z)\big)=z.\]

(5) 由 \ref{thm9.6.2.3} 知$\varphi_a$是把$B_n$映入$B_n$的全纯映射,由 \ref{thm9.6.2.4} 知$\varphi_a^{-1}=\varphi_a$是全纯的,因而$\varphi_a\in\Aut(B_n)$.
\end{proof}

上面证明了$\varphi_a$是$B_n$的一个全纯自同构,但是否$B_n$的每个自同构都能写成这种样子呢?下面的定理断言,$B_n$的任何自同构必是$\varphi$和一个酉变换的复合.
\begin{theorem}\label{thm9.6.3}
设$\psi\in\Aut(B_n)$,如果$\psi^{-1}(0)=a$,则必存在唯一的酉方阵$U$,使得对每个$z\in B_n$,有
\[\psi(z)=\varphi_a(z)U.\]
\end{theorem}
\begin{proof}
记$f=\psi\circ \varphi_a$,则$f\in\Aut(B_n)$,且$f(0)=\psi(a)=0$. 由定理 \ref{thm9.6.1} 知,$f$是一个酉变换,即存在酉方阵$U$,使得$\psi\big(\varphi_a(w)\big)=wU$. 令$\varphi_a(w)=z$,则$w=\varphi_a(z)$,因而$\psi(z)=\varphi_a(z)U$. $U$的唯一性是显然的.
\end{proof}

现在很容易证明下面的
\begin{theorem}\label{thm9.6.4}
单位球$B_n$是$\MC^n$中的可递域.
\end{theorem}
\begin{proof}
任取$a,b\in B_n$,则有$\varphi_a\in\Aut(B_n)$,使得$\varphi_a(a)=0$,同时有$\varphi_b\in\Aut(B_n)$,使得$\varphi_b(0)=b$. 令$\psi=\varphi_b\circ\varphi_a$,则当然$\psi\in\Aut(B_n)$,而$\psi(a)=b$.所以$B_n$是可递域.
\end{proof}

\begin{definition}\label{def9.6.5}
设$\Omega_1$和$\Omega_2$是$\MC^n$中的两个域,如果存在双全纯映射把$\Omega_1$映为$\Omega_2$,就称$\Omega_1$和$\Omega_2$是\textbf{全纯等价}\index{Q!全纯等价}的.
\end{definition}

在单复变中,Riemann定理断言,除了整个复平面$\MC$以外,$\MC$上任意两个单连通域都是全纯等价的.在多复变中,Riemann定理不再成立,即使是两个最简单的域$U^n$和$B_n$也不是全纯等价的.这一事实于本世纪初首先由Poincar\'e指出.
\begin{theorem}[(\textbf{Poincar\'e})]
多圆柱$U^n$和球$B_n$不全纯等价.
\end{theorem}
\begin{proof}
如果$U^n$和$B_n$全纯等价,那么存在双全纯映射$F$,使得$F(B_n)=U^n$.由于$B_n$是圆型的,而且是有界可递的,$U^n$也是圆型的,由定理 \ref{thm9.5.6},一定存在线性映射把$B_n$映为$U^n$.但线性映射把球映为椭球,不可能是多圆柱.这个矛盾证明了$U^n$和$B_n$不是全纯等价的.
\end{proof}

\begin{xiti}
\item 对于任意$z,w\in B_n$,证明:
\[|\varphi_z(w)|=|\varphi_w(z)|.\]
\item 设$a,z,w\in B_n$,证明:
\[|\varphi_{\varphi_a(w)}(z)|=\big|\varphi_w\big(\varphi_a(z)\big)\big|.\]
(\textbf{提示}:考虑映射$\varphi_{\varphi_a(w)}\circ\varphi_a\circ\varphi_w$.)
\item 定义$E(a,r)=\varphi_a^{-1}\big(B(0,r)\big)=\{z:|\varphi_a(z)|<r\}$. 证明:对任意$a,z\in B_n,0<r<1$,有
    \[\varphi_a\big(E(z,r)\big)=E\big(\varphi_a(z),r\big).\]
\item 设$F\in H(B_n),F=f\circ\varphi_z$. 证明:
\begin{enuma}
  \item $(\nabla F)(0)=(\nabla f)(z)\varphi_z'(0)$,这里
  \[(\nabla f)(z)=\bigg(\pp{f(z)}{z_1},\cdots,\pp{f(z)}{z_n}\bigg);\]
  \item $|(\nabla F)(0)|^2=(1-|z|^2)[|(\nabla F)(z)|^2-|(\textrm Rf)(z)|^2]$,这里,$(\textrm Rf)(z)=\sum_{j=1}^nz_j\pp{f(z)}{z_j}$是$f$的径向导数.
\end{enuma}
\item 设$\psi\in\Aut(B_n)$,如果$\psi(a)=0$,那么对任意$z,w\in\bar{B_n}$,有等式
\[1-\psi(z)\bar{\psi(w)'}=\frac{(1-|a|^2)(1-z\bar{w'})}
{(1-\bar az')(1-a\bar {w'})}.\]
\item 设$\psi\in\Aut(B_n)$. 如果$\psi$在$\partial B_n$上有三个不同的不动点,证明$\psi$在$B_n$中至少有一个不动点.\\
(\textbf{提示}:设$\zeta_1,\zeta_2,\zeta_3$是$\psi$在$\partial B_n$上的三个不动点,令$z_0=\frac12(\zeta_1+\zeta_2)$,先证明$\varphi_a(z_0)=\frac12\big(\varphi_a(\zeta_1)+\varphi_a(\zeta_2)\big)$,这里$a=\psi^{-1}(0)$,然后再证明$\psi(z_0)=z_0$.)
\end{xiti}
